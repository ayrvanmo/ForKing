\section{Conclusiones}

A raíz de la realización de este proyecto se pudo comprender de mejor manera los usos reales de las estructuras de datos analizadas durante las clases. Además fue posible evidenciar la importancia de los conceptos introducidos por estas estructuras para el desarrollo de nuevas estructuras como el caso del \textit{BuddySystem}.

Puesto que este proyecto permitió el trabajo con grandes volúmenes de datos se comprendió de mejor manera los conceptos de eficiencia de algoritmos como el \texttt{Merge Sort} en contra de los algoritmos de ordenamiento de listas como \texttt{Bubble Sort}. Así como la importancia de la creación de funciones eficientes para el manejo de las diferentes estructuras de datos.

Además, al tratarse de una simulación de un gestor de procesos y de memoria se pudo comprender mejor el funcionamiento de algoritmos como \texttt{Round Robin} y \texttt{Shortest Job First}, los cuales son usados en muchos de los sistemas operativos modernos.