\section{Posibles Mejoras a futuro}
\rhostart{A} pesar de su correcta funcionalidad \textit{forKing} tiene algunas limitaciones que podrían ser mejoradas en el futuro así como caracteristicas que podrían mejorar su funcionalidad pero que, por distintas razones, fueron omitidas; estas son las siguientes:
\begin{enumerate}
    \item \textbf{Ordenamiento de colas}: La cola \texttt{sjfQueue}, para su correcto funcionamiento debe estar ordenada de tal manera que el proceso que se encuentra al inicio de la misma sea aquel de la cola con el menor \texttt{burstTime}; este es también el caso de la cola \texttt{WaitingQueue} donde el primer proceso debe ser el de menor requerimiento de memoria.
    El proceso usado para conseguir este resultado tiene una complejidad de $O(n)$, sin embargo no se encontró una manera más eficiente de conseguirlo.
    Lo anterior ocasiona que para cantidades de procesos muy grandes la simulación pueda tardar mucho tiempo en ejecutarse.
    \item \textbf{Uso ineficiente del BuddySystem}: El \textit{BuddySystem} a pesar de solucionar la mayoría de problemas de asignación de memoria tiene un defecto clave: Depende de espacios con potencias de dos, esto ocasiona que si un proceso requiere $1025$ bytes de memoria, el \textit{BuddySystem} tendrá que asignar $2048$ bytes de memoria a dicho proceso, desperdiciando $1023$ bytes de memoria.
    Esto podría solucionarse complejizando más el algoritmo de asignación de memoria.
    \item \textbf{Trabajo con hebras y/o procesos paralelos}: Una de las ideas más ambiciosas surgida en los inicios del desarrollo de forKing fue el trabajar con procesos paralelos, mediante el uso de \textit{hebras}\footnote{Unidad más pequeña de ejecución dentro de un proceso, que permite realizar tareas concurrentes compartiendo la memoria del proceso principal} o, de procesos generados con \texttt{fork()}\footnote{Funcion estandar de la libreria ``unistd.h'', la cual permite generar procesos hijos. \parencite{fork_function}} (de ahí el nombre del proyecto), sin embargo dada la complejidad de esta implementación y el tiempo con el que se contaba para la realización del proyecto se prefirió no enforcar los esfuerzos en dicha dirección.
    \item \textbf{Bloqueo de procesos, espera y ráfagas de E/S}: Durante el inicio de este proyecto se realizó una simplificación de lo que un simulador de un gestor de memoria y planificación de procesos debería hacer, en esta etapa se simplificó su funcionamiento dejando de lado algunas características como:
    \begin{itemize}
        \item \textbf{Bloqueo de procesos}: Se decidió no tener en cuenta el bloqueo de procesos, donde un proceso podría necesitar de la previa ejecución de otro para proceder con su ejecución.
        \item \textbf{Ráfagas de E/S}: En sistemas operativos reales un proceso no necesita un único \textit   {burstTime} para su completa ejecución sino que puede requerir de varias ráfagas de CPU combinadas con ráfagas de E/S, por lo que se decidió no tener en cuenta esta característica.
    \end{itemize}
\end{enumerate}















